\documentclass[12pt]{article}
 
\usepackage[margin=1in]{geometry} 
\usepackage{amsmath,amsthm,amssymb}
\usepackage{float}
\usepackage{tikz}
\usetikzlibrary{arrows,shapes,trees} % loads some tikz extensions
 
\begin{document}
 
\title{Homework 2}
\author{Josh Klontz
CSE 802}
 
\maketitle

\begin{enumerate}
\item \textbf{Solve the following problems from Chapter 2 of the textbook: 8, 10, 14 Parts (a), (b), and (c), 24}
  \subitem \textbf{8. Let the conditional densities for a two-category one-dimensional problem be given by the Cauchy distribution described in Problem 7.}
  \begin{enumerate}
  \item \textbf{By explicit integration, check that the distributions are indeed normalized.}
    \begin{figure}[H]
    \begin{equation}
      p(x|w_i) = \frac{1}{\pi b} \cdot \frac{1}{1 + (\frac{x-a_i}{b})^2}
    \end{equation}
    \caption{Probability density function of a Cauchy distribution.}
    \end{figure}
    The probability density is considered to be normalized if it integrates to 1:
    \begin{figure}[H]
    \begin{equation}
    \begin{split}
      \int_{-\infty}^{\infty} p(x|w_i) dx& = \int_{-\infty}^{\infty} \frac{1}{\pi b} \cdot \frac{1}{1 + (\frac{x-a_i}{b})^2} dx \\
      & = \frac{1}{\pi b} \int_{-\infty}^{\infty} \frac{1}{1 + (\frac{x-a_i}{b})^2} dx
    \end{split}
    \end{equation}
    Apply U-substituion:
    \begin{equation}
    \begin{split}
      u& = \frac{x-a_i}{b} \\
      & = \frac{x}{b} \cdot \frac{a_i}{b} \\
    \frac{du}{dx}& = \frac{1}{b} \\
    dx& = b\ du
    \end{split}
    \end{equation}
    To obtain:
    \begin{equation}
    \begin{split}
      \int_{-\infty}^{\infty} p(x|w_i) dx& = \frac{1}{\pi} \int_{-\infty}^{\infty} \frac{1}{1 + u^2} du \\
      & = \frac{1}{\pi} \cdot tan^{-1}(u) |_{-\infty}^{\infty} \\
      & = \frac{1}{\pi} (\frac{\pi}{2} - \frac{-\pi}{2}) \\
      & = \frac{1}{\pi} \pi \\
      & \boxed{= 1}
    \end{split}
    \end{equation}
    \end{figure}
  \item \textbf{Assuming that $P(w_1) = P(w_2)$, show that $P(w_1|x) = P(w_2|x)$ if $x = (a_1 + a_2)/2$.}
    \begin{figure}[H]
    \begin{equation}
    \begin{split}
      P(w_1|x)& \iff P(w_2|x) \\
      \frac{p(x|w_1)P(w_1)}{p(x)}& \iff \frac{p(x|w_2)P(w_2)}{p(x)} \\
      p(x|w_1)P(w_1)& \iff p(x|w_2)P(w_2) \\
      p(x|w_1)& \iff p(x|w_2) \\
      \frac{1}{\pi b} \cdot \frac{1}{1 + (\frac{x-a_1}{b})^2}& \iff \frac{1}{\pi b} \cdot \frac{1}{1 + (\frac{x-a_2}{b})^2} \\
      \frac{1}{1 + (\frac{x-a_1}{b})^2}& \iff \frac{1}{1 + (\frac{x-a_2}{b})^2} \\
      1 + (\frac{x-a_1}{b})^2& \iff 1 + (\frac{x-a_2}{b})^2 \\
      (\frac{x-a_1}{b})^2& \iff (\frac{x-a_2}{b})^2 \\
      \frac{x^2-2xa_1+a_1^2}{b^2}& \iff \frac{x^2-2xa_2+a_2^2}{b^2} \\
      x^2-2xa_1+a_1^2& \iff x^2-2xa_2+a_2^2 \\
      -2xa_1+a_1^2& \iff -2xa_2+a_2^2 \\
      -2(\frac{a_1+a_2}{2})a_1 + a_1^2& \iff -2(\frac{a_1+a_2}{2})a_2 + a_2^2 \\
      -(a_1+a_2)a_1 + a_1^2& \iff -(a_1+a_2)a_2 + a_2^2 \\
      (a_1+a_2)a_2 + a_1^2& \iff (a_1+a_2)a_1 + a_2^2 \\
      a_1^2+ a_1a_2 + a_2^2 & = a_1^2+ a_1a_2 + a_2^2 \\
      & \qed
    \end{split}
    \end{equation}
    \end{figure}
  \item \textbf{Plot $P(w_1|x)$ for the case $a_1=3, a_2=5, b=1$.}
    \begin{figure}[H]
    \begin{equation}
    \begin{split}
      P(w_1|x)& = \frac{p(x|w_1)P(w_1)}{p(x)} \\
      & = \frac{\frac{1}{\pi b} \cdot \frac{1}{1 + (\frac{x-a_1}{b})^2} \cdot 0.5}{\frac{1}{\pi b} \cdot \frac{1}{1 + (\frac{x-a_1}{b})^2} \cdot 0.5 + \frac{1}{\pi b} \cdot \frac{1}{1 + (\frac{x-a_2}{b})^2} \cdot 0.5} \\
      & = \frac{\frac{1}{1 + (\frac{x-a_1}{b})^2}}{\frac{1}{1 + (\frac{x-a_1}{b})^2} + \frac{1}{1 + (\frac{x-a_2}{b})^2}} \\
      & = \frac{1 + (\frac{x-a_2}{b})^2}{1 + (\frac{x-a_1}{b})^2 + 1 + (\frac{x-a_2}{b})^2} \\
      & = \frac{1 + (x-a_2)^2}{2b^2 + (x-a_1)^2 + (x-a_2)^2} \\
      & = \frac{b^2 + (x-a_2)^2}{2b^2 + (x-a_1)^2 + (x-a_2)^2} \\
      & = \frac{1 + (x-5)^2}{2 + (x-3)^2 + (x-5)^2} \\
      & = \frac{x^2 - 10x + 26}{2x^2-16x+36}
    \end{split}
    \end{equation}
    \includegraphics[width=\textwidth]{8c}
    \end{figure}
  \item \textbf{How do $P(w_1|x)$ and $P(w_2|x)$ behave as $x \rightarrow -\infty$ and $x \rightarrow +\infty$? Explain.}
    $P(w_1|x)$ and $P(w_2|x)$ approach $\frac{1}{2}$. This is because they have the same variance and their difference in means is decreasingly important as $x \rightarrow \pm \infty$.
  \end{enumerate}

\end{enumerate}
 
\end{document}
