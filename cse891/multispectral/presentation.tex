% A Readymade beamer presentation template
% Version 1.1
% Relase date: May 2, 2010
% Released at http://www.stattler.com
% by Rifat Jahan

\documentclass{beamer}
%\usecolortheme[named=green]{structure}
\mode<presentation> {
\usetheme{Madrid} % My favorite!
%\usetheme{Boadilla} % Pretty neat, soft color.
%\usetheme{default}
%\usetheme{Warsaw}
%\usetheme{Bergen} % This template has nagivation on the left
%\usetheme{Frankfurt} % Similar to the default with an extra region at the top.
%\usecolortheme{seahorse} % Simple and clean template
%\usetheme{Darmstadt} % not so good
% Uncomment the following line if you want page numbers and using Warsaw theme
% \setbeamertemplate{footline}[page number]
%\setbeamercovered{transparent}
\setbeamercovered{invisible}
% To remove the navigation symbols from the bottom of slides%
\setbeamertemplate{navigation symbols}{} 
}

\usepackage{graphicx}
%\usepackage{bm} 
% For typesetting bold math (not \mathbold)
%\logo{\includegraphics[height=0.6cm]{yourlogo.eps}}

\title[Hyperspectral]{Hyperspectral Face Recognition}

\author{Josh Klontz}
%\institute[U of X]
%{
%University of [...] \\
%\medskip
%{\emph{email@domain.ca}}
%}
\date{\today}
% \today will show current date. 
% Alternatively, you can specify a date.

\begin{document}

\begin{frame}
\titlepage
\end{frame}

\section{Introdution}
\begin{frame}
\frametitle{Works Covered}
\begin{block}{Face recognition in hyperspectral images}
Pan, Zhihong, et al. "Face recognition in hyperspectral images." Pattern Analysis and Machine Intelligence, IEEE Transactions on 25.12 (2003): 1552-1560.
\end{block}
\pause
\begin{block}{Face recognition by fusing thermal infrared and visible imagery}
Bebis, George, et al. "Face recognition by fusing thermal infrared and visible imagery." Image and Vision Computing 24.7 (2006): 727-742.
\end{block}
\pause
\begin{block}{Heterogeneous face recognition using kernel prototype similarities}
Klare, Brendan, and A. Jain. "Heterogeneous face recognition using kernel prototype similarities." (2012): 1-1.
\end{block}
\end{frame}

\section{Hyperspectral}
\begin{frame}
\frametitle{Hyperspectral}
\begin{block}{Abstract}
\begin{itemize}
\item 31 channel NIR acquisition
\item Model the subsurface tissue structure
\item More invariant to pose and expression
\end{itemize}
\end{block}
\end{frame}

\begin{frame}
\centerline{The End!}
\end{frame}

% End of slides
\end{document}