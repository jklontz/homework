\documentclass[12pt]{article}
 
\usepackage[margin=1in]{geometry} 
\usepackage{amsmath,amsthm,amssymb}
 
\begin{document}
 
\title{Week 10}
\author{Josh Klontz\\CSE 891}
 
\maketitle
 
The papers this week explore various approaches to face recognition under variable pose and illumination.
The first is the comprehensive review paper by Zhang and Gao titled ``Face Recognition across pose: A review'' which covers a broad collection of recently proposed approaches to pose invariant face recognition including general methods, 2D techniques, and 3D models.
The second paper, ``Locally Linear Regression for Pose-Invariant Face Recognition'' by Chai et al., considers pose correction by a collection of linear mappings of local patches on the face to a frontal pose.
The third work by Gross, Matthews, and Baker on ``Eigen Light-Fields and Face Recognition Across Pose'', proposes modeling the light-field of a head in a PCA subspace.
The final paper, ``An efficient illumination normalization method for face recognition", by Xie and Lam introduces a local block-based illumination correction technique as a preprocessing step for eigenface-like algorithms.
\par
The review paper of Zhang and Gao is the most extensive survey of face recognition across pose to date.
The paper discusses three major classes of algorithmic approaches including general face recognition techniques, 2D pose invariant techniques, and 3D model based algorithms.
General approaches are further classified as either wholistic or local, with local approaches being slightly more tolerant to misalignment though neither exhibit strong pose invariance.
A wide variety of algorithms are considered in the 2D pose invariant category including real view-based matching where gallery templates are provided across varying pose so as to improve likelihood of matching similarly posed templates.
Numerous warping techniques are considered both in image space and in feature space and generally exhibit strong matching capability across minor pose changes but break down under large pose variation.
3D models are more promising for matching across dramatic viewpoint variation though they suffer from both increased algorithmic and computational complexity.
\par
The fundamental contribution of the work done by Chai et al.\ is an effective approach for minor 2D pose correction.
The authors demonstrate success using patch based regression to warp to a frontal pose.
The algorithm is validated on the CMU PIE database and exhibits compelling visual performance.
\par
One of the earlier approaches to pose invariant face recognition is the work by Gross et al.\ using eigen light-fields.
The idea is that a face, like any object, can be modeled in a 4D space that includes all possible viewpoints.
Thie light-field space is intractably large, so the authors propose modeling it in a lower dimensional eigen subspace.
Nearest neighbor matching is done within this eigen light-field space.
While the approach is shown to work well in controlled enviorments, the authors point out that a lot of work remains to tackle the problem in a fully automated uncontrolled environment.
\par
Unlike the other papers reviewed, the last work by Xie and Lam focuses on illumination correction.
The key contribution of their work is a local normalization approach whereby patches are normalized to zero mean and unit variance. 
The approach is simple, computationally efficient, and provides a moderately effective preprocessing step.
\end{document}
